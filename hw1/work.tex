\documentclass[12pt]{article}

\usepackage{amsmath}
\usepackage{amssymb}
\usepackage{algpseudocode}

\pagestyle{myheadings}
\markright{CSE 190 Homework 1 \hfill Matthias Springer, A99500782\hfill}

\begin{document}

\section*{Problem 2}

In the next subsections, we refer to the constraints with the following numbers.

\begin{enumerate}
	\item $-5x - 4y - 3z -2w \geq -1$
	\item $x \geq 0$
	\item $y \geq 0$
	\item $z \geq 0$
	\item $w \geq 0$
\end{enumerate}

\subsection*{Feasible region is bounded}
Constraints 2-5 ensure that all variables are positive. The first constraint can be written as follows.

$$ 5x+4y+3z+2w \leq 1 $$

Therefore, $x \leq \frac{1}{5}$, $y \leq \frac{1}{4}$, $z \leq \frac{1}{3}$, and $w \leq \frac{1}{2}$. A bigger value of $x$, $y$, $z$, or $w$ would violate the first constraint, regardless of the other variables' values, as long as they are positive.

We show that the feasible region is bounded by arguing that the whole feasible regions fits inside a ball of a fixed radius, where the origin is the center of the ball. Consider $x=\frac{1}{5}$, $y=\frac{1}{4}$, $z=\frac{1}{3}$, and $w=\frac{1}{3}$. Since these are the maximum values for the variables, a point can never be further away from the center. The distance from the center is $\sqrt{x^2 + y^2 + z^2 + w^2} < 0.7$. Therefore, the whole feasible region is enclosed in a ball if radius $0.7$ and thus bounded.

\subsection*{Find extreme point candidates}
\begin{itemize}
	\item Constraints 2, 3, 4, 5 \\ $\Rightarrow x=0, y=0, z=0, w=0$
	\item Constraints 1, 3, 4, 5 \\ $\Rightarrow x=\frac{1}{5}, y=0, z=0, w=0$
	\item Constraints 1, 2, 4, 5 \\ $\Rightarrow x=0, y=\frac{1}{4}, z=0, w=0$
	\item Constraints 1, 2, 3, 5 \\ $\Rightarrow x=0, y=0, z=\frac{1}{3}, w=0$
	\item Constraints 1, 2, 3, 4 \\ $\Rightarrow x=0, y=0, z=0, w=\frac{1}{2}$
\end{itemize}

\subsection*{Solve linear problem}
We evaluate the objective function at all extreme point candidates (in the same order). The solution of the linear problem is the point that minimizes the objective function.
\begin{itemize}
	\item $0 - 2 \cdot 0 + 3 \cdot 0 - 4 \cdot 0 = 0$
	\item $\frac{1}{5} - 2 \cdot 0 + 3 \cdot 0 - 4 \cdot 0 = \frac{1}{5}$
	\item $0 - 2 \cdot \frac{1}{4} + 3 \cdot 0 - 4 \cdot 0 = -\frac{1}{2}$
	\item $0 - 2 \cdot 0 + 3 \cdot \frac{1}{3} - 4 \cdot 0 = 1$
	\item $0 - 2 \cdot 0 + 3 \cdot 0 - 4 \cdot \frac{1}{2} = -2$
\end{itemize}

The last extreme point candidate minimizes the objective function, so $x=0, y=0, z=0, w=\frac{1}{2}$ is the solution of the linear problem.

\section*{Problem 3}
In the next subsections, we refer to the constraints with the following numbers.

\begin{enumerate}
	\item $x + 2y + 3z + 4w \geq 5$
	\item $-y -2w \geq -1$
	\item $x \geq 0$
	\item $y \geq 0$
	\item $z \geq 0$
	\item $w \geq 0$
\end{enumerate}

\subsection*{Feasible region is unbounded}
Constraints 3-6 ensure that all variables are positive. The second constraint can be written as follows.

$$y + 2w \leq 1$$

We show that the feasible region is unbounded by arguing that the whole feasible region can never fit inside a ball of a fixed radius.

$x=5$, $y=0$, $z=0$, $w=0$ is a point inside the feasible region, because it satisfies all constraints. We can now increase $z$ arbitrarily and all constraints will stay satisfied: in constraint 1, the sum will get even bigger; in constraint 5, $z$ will remain positive; and the other constraints do not constrain $z$ at all.

The feasible region can never fit inside a ball of a fixed size, regardless of the center of the ball. We can always find a point with a bigger value of $z$ that is outside the ball. Therefore, the feasible region is unbounded.

\subsection*{Find extreme point candidates}
\begin{itemize}
	\item Constraints 3, 4, 5, 6 \\ $\Rightarrow x=0, y=0, z=0, w=0$
	\item Constraints 2, 3, 4, 5 \\ $\Rightarrow x=0, y=0, z=0, w=\frac{1}{2}$
	\item Constraints 2, 4, 5, 6 \\ no solution
	\item Constraints 2, 3, 5, 6 \\ $\Rightarrow x=0, y=1, z=0, w=0$
	\item Constraints 2, 3, 4, 6 \\ no solution
	\item Constraints 1, 3, 4, 5 \\ $\Rightarrow x=0, y=0, z=0, w=\frac{5}{4}$
	\item Constraints 1, 4, 5, 6 \\ $\Rightarrow x=5, y=0, z=0, w=0$
	\item Constraints 1, 3, 5, 6 \\ $\Rightarrow x=0, y=\frac{5}{2}, z=0, w=0$
	\item Constraints 1, 3, 4, 6 \\ $\Rightarrow x=0, y=0, z=\frac{5}{3}, w=0$
	\item Constraints 1, 2, 3, 4 \\ $\Rightarrow x=0, y=0, w=\frac{1}{2}, z=1$
	\item Constraints 1, 2, 3, 5 \\ no solution
	\item Constraints 1, 2, 3, 6 \\ $\Rightarrow x=0, y=1, z=1, w=0$
	\item Constraints 1, 2, 4, 6 \\ no solution
	\item Constraints 1, 2, 4, 5 \\ $\Rightarrow x=3, y=0, z=0, w=\frac{1}{2}$
	\item Constraints 1, 2, 5, 6 \\ $\Rightarrow x=3, y=1, z=0, w=0$
\end{itemize}

\subsection*{Solve linear problem}
We evaluate the objective function at all extreme point candidates (in the same order) and check if all constraints are satisfied. The solution of the problem is the point that minimizes the objective function.

\begin{itemize}
	\item Constraint 1 violation
	\item Constraint 1 violation
	\item Linear equations had no solution
	\item Constraint 1 violation
	\item Linear equations had no solution
	\item Constraint 2 violation
	\item 5
	\item Constraint 2 violation
	\item 5
	\item 1
	\item Linear equations had no solution
	\item 1
	\item Linear equations had no solution
	\item 1
	\item 1
\end{itemize}

$x=0, z=1, w=\frac{1}{2}, y=0$ minimizes the objective function, is within the feasible area, and is therefore the solution of this linear program. As we can see in the previous list, this is not the only solution.

\section*{Problem 4}
This is the basic idea: we want to maximize the profit, i.e. the sales revenue minus the production costs. A radio can be sold for a certain ammount of dollars, depending on the week. In the objective function, the 5\$ production cost for every radio is already substracted from the revenue per radio.

We distinguish three types of persons. Trainees get 100\$ per week and cannot produce radios. Workers get 200\$ per week and can either produce up to 50 radios per week or educate up to 3 trainees per week (instructors). Therefore, the number of instuctors limits the number of trainees. It is possible that a trainee is hired in one week and fired in the next week, without even having worked. However, this will never happen when we maximize the profit.

$ \mbox{Maximize } 15 \cdot \mathit{radio1} + 13 \cdot \mathit{radio2} + 11 \cdot \mathit{radio3} + 9 \cdot \mathit{radio4} - 200 \cdot \mathit{workers1} - 200 \cdot \mathit{workers2} - 200 \cdot \mathit{workers3} - 200 \cdot \mathit{workers4} - 100 \cdot \mathit{trainees1} - 100 \cdot \mathit{trainees2} - 100 \cdot \mathit{trainees3} $ subject to

$$ \mathit{radio1}, \mathit{radio2}, \mathit{radio3}, \mathit{radio4} \geq 0 $$
$$ \mathit{workers1}, \mathit{workers2}, \mathit{workers3}, \mathit{workers4} \geq 0$$
$$ \mathit{trainees1}, \mathit{trainees2}, \mathit{trainees3} \geq 0 $$
$$ \mathit{instructors1}, \mathit{instructors2}, \mathit{instructors3} \geq 0 $$

$$ \mathit{radio1} \leq 50 \cdot \mathit{workers1} - 50 \cdot \mathit{instructors1} $$
$$ \mathit{radio2} \leq 50 \cdot \mathit{workers2} - 50 \cdot \mathit{instructors2} $$
$$ \mathit{radio3} \leq 50 \cdot \mathit{workers3} - 50 \cdot \mathit{instrcutors3} $$
$$ \mathit{radio4} \leq 50 \cdot \mathit{workers4} $$

$$ \mathit{workers1} = 40 $$
$$ \mathit{workers2} \leq \mathit{workers1} + \mathit{trainees1} $$
$$ \mathit{workers3} \leq \mathit{workers2} + \mathit{trainees2} $$
$$ \mathit{workers4} \leq \mathit{workers3} + \mathit{trainees3} $$

$$ \mathit{instructors1} \leq \mathit{workers1} $$
$$ \mathit{instructors2} \leq \mathit{workers2} $$
$$ \mathit{instructors3} \leq \mathit{workers3} $$

$$ \mathit{trainees1} \leq 3 \cdot \mathit{workers1} $$
$$ \mathit{trainees2} \leq 3 \cdot \mathit{workers2} $$
$$ \mathit{trainees3} \leq 3 \cdot \mathit{workers3} $$

$$ \mathit{radio1} + \mathit{radio2} + \mathit{radio3} + \mathit{radio4} = 20000 $$

All equality constraints can be converted into two unequality constraints with $\leq$ and $\geq$ relations. All variables should be integer variables, since we cannot hire a fraction of a worker/trainee or produce fractions of radios.


\section*{Problem 5}
We use the following letters as shortcuts for the rolls of different width.

\begin{enumerate}
	\item A: 68cm
	\item B: 173cm
	\item C: 148cm
	\item D: 127cm
\end{enumerate}

The idea of my solution is to list all possible ways to divide a 400cm roll into smaller rolls. For example, it could be divided in AAAAA or CC. It is not possible, to add another roll to these two examples, because then the width would exceed 400cm.

In total, there are 12 possibilites to divide a 400cm roll: AAAAA, AAAAD, AAAB, AAAC, AADD, ACC, ABC, ABD, ACD, BB, CC, DDD. I calculated this list of divisions using this idea: starting with AAAAA, BB, CC, and DDD (the maximum number of equal rolls), remove a roll and try to add rolls with different sizes.

\begin{itemize}
	\item $ \mbox{Minimize } n_{AAAAA} + n_{AAAAD} + n_{AAAB} + n_{AAAC} + n_{AADD} + n_{ACC} + n_{ABC} + n_{ABD} + n_{ACD} + n_{BB} + n_{CC} + n_{DDD} \mbox{ subject to}$
	\item $ 5 n_{AAAAA} + 4 n_{AAAAD} + 3 n_{AAAB} + 3 n_{AAAC} + 2 n_{AADD} + n_{ACC} + n_{ABC} + n_{ABD} + n_{ACD} \geq 106$
	\item $ n_{AAAB} + n_{ABC} + n_{ABD} + 2 n_{BB} \geq 28$
	\item $ n_{AAAC} + 2 n_{ACC} + n_{ABC} + n_{ACD} + 2 n_{CC} \geq 205$
	\item $ n_{AAAAD} + 2 n_{AADD} + n_{ABD} + n_{ACD} + 3 n_{DDD} \geq 93$
\end{itemize}

All variables should be integers (integer linear progamming), because we are only allowed to use full rolls. Otherwise, the LP solver could, for instance, use only $\frac{1}{5}$ of $n_{AAAAA}$ if one A is missing at the end. \footnote{This can even happen, if we add divisions that do not fill up all the 400cm (e.g. AAA): the solver could decide to use half of ACC and half of A instead of a full AC.}
\end{document}

